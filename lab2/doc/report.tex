\documentclass[a4paper, 12pt]{article}
\usepackage{cmap}
\usepackage[12pt]{extsizes}			
\usepackage{mathtext} 				
\usepackage[T2A]{fontenc}			
\usepackage[utf8]{inputenc}			
\usepackage[english,russian]{babel}
\usepackage{setspace}
\singlespacing
\usepackage{amsmath,amsfonts,amssymb,amsthm,mathtools}
\usepackage{fancyhdr}
\usepackage{soulutf8}
\usepackage{euscript}
\usepackage{mathrsfs}
\usepackage{listings}

\usepackage[colorlinks=true, urlcolor=blue, linkcolor=black]{hyperref}

\pagestyle{fancy}
\usepackage{indentfirst}
\usepackage[top=10mm]{geometry}
\rhead{}
\lhead{}
\renewcommand{\headrulewidth}{0mm}
\usepackage{tocloft}
\renewcommand{\cftsecleader}{\cftdotfill{\cftdotsep}}
\usepackage[dvipsnames]{xcolor}

\lstdefinestyle{mystyle}{ 
	keywordstyle=\color{OliveGreen},
	numberstyle=\tiny\color{Gray},
	stringstyle=\color{BurntOrange},
	basicstyle=\ttfamily\footnotesize,
	breakatwhitespace=false,         
	breaklines=true,                 
	captionpos=b,                    
	keepspaces=true,                 
	numbers=left,                    
	numbersep=5pt,                  
	showspaces=false,                
	showstringspaces=false,
	showtabs=false,                  
	tabsize=2
}

\lstset{style=mystyle}

\begin{document}
\thispagestyle{empty}	
\begin{center}
	Московский авиационный институт
	
	(Национальный исследовательский университет)
	
	Факультет "Информационные технологии и прикладная математика"
	
	Кафедра "Вычислительная математика и программирование"
	
\end{center}
\vspace{40ex}
\begin{center}
	\textbf{\large{Лабораторная работа №2 по курсу\linebreak \textquotedblleft Операционные системы\textquotedblright}}
\end{center}
\vspace{35ex}
\begin{flushright}
	\textit{Студент: } Иванов Андрей Кириллович
	
	\vspace{2ex}
	\textit{Группа: } М8О-208Б-22
	
	\vspace{2ex}
	\textit{Преподаватель: } Миронов Евгений Сергеевич
	
	\vspace{2ex}
	\textit{Вариант: } 17
	
	\vspace{2ex}
	\textit{Оценка: } \underline{\quad\quad\quad\quad\quad\quad}
	
	 \vspace{2ex}
	\textit{Дата: } \underline{\quad\quad\quad\quad\quad\quad}
	
	\vspace{2ex}
	\textit{Подпись: } \underline{\quad\quad\quad\quad\quad\quad}
	
\end{flushright}

\vspace{5ex}

\begin{vfill}
	\begin{center}
		Москва, 2023
	\end{center}	
\end{vfill}
\newpage


\begingroup
\color{black}
\tableofcontents\newpage
\endgroup

\section{Репозиторий}
\href{https://github.com/kumaroid/osLabs}{https://github.com/kumaroid/osLabs}

\section{Цель работы}
Приобретение практических навыков в:
\begin{itemize}
    \item Управлении потоками в ОС
    \item Обеспечении синхронизации между потоками
\end{itemize}

\section{Задание}
Составить программу на языке Си, обрабатывающую данные в многопоточном режиме. При обработки использовать стандартные средства создания потоков операционной системы (Windows/Unix). Ограничение максимального количества потоков, работающих в один момент времени, должно быть задано ключом запуска вашей программы.

\section{Описание работы программы}
Отсортировать массив целых чисел при помощи TimSort

В ходе выполнения лабораторной работы я использовал следующие системные вызовы:
\begin{itemize}
    \item pthread\_create() - создание потока
    \item pthread\_join() - ожидание завершения потока
\end{itemize}


\newpage

\section{Исходный код}
timsort.hpp
\begin{lstlisting}[language=C++]
#pragma once
#include <bits/stdc++.h>

void timSort(int *arr, int n, int maxThreads);
void printArray(int arr[], int n);
\end{lstlisting}

timsort.cpp
\begin{lstlisting}[language=C++]
#include "timsort.hpp"
#include <cmath>

const int RUN = 4;

struct Args1 {
    int* arr;
    int left;
    int right;
};

struct Args2 {
    int* arr;
    int l;
    int m;
    int r;
};

void printArray(int arr[], int n)
{
    for (int i = 0; i < n; i++)
        printf("%d  ", arr[i]);
    printf("\n");
}

void* insertionSort(void* args)
{
    const auto &ar = *(Args1*)args;
    int *arr = ar.arr;
    int left = ar.left;
    int right = ar.right;
    for (int i = left + 1; i <= right; i++) {
        int temp = arr[i];
        int j = i - 1;
        while (j >= left && arr[j] > temp) {
            arr[j + 1] = arr[j];
            j--;
        }
        arr[j + 1] = temp;
    }

    // sleep(50);
    pthread_exit(0);
}

void* merge(void* args)
{
    Args2 arg = *(Args2*)args;
    int* arr = arg.arr;
    int m = arg.m;
    int r = arg.r;
    int l = arg.l;

    int len1 = m - l + 1, len2 = r - m;
    int left[len1], right[len2];
    for (int i = 0; i < len1; i++)
        left[i] = arr[l + i];
    for (int i = 0; i < len2; i++)
        right[i] = arr[m + 1 + i];

    int i = 0;
    int j = 0;
    int k = l;

    while (i < len1 && j < len2) {
        if (left[i] <= right[j]) {
            arr[k] = left[i];
            i++;
        }
        else {
            arr[k] = right[j];
            j++;
        }
        k++;
    }

    while (i < len1) {
        arr[k] = left[i];
        k++;
        i++;
    }

    while (j < len2) {
        arr[k] = right[j];
        k++;
        j++;
    }

    // sleep(50);
    pthread_exit(0);
}

void timSort(int *arr, int n, int maxThreads)
{

    int threadCount = std::min(n / RUN + 1, maxThreads);
    if (maxThreads == -1) {
        threadCount = n / RUN + 1;
    }
    pthread_t tid[threadCount];
    int j = 0;
    for (int i = 0; i < n; i += RUN)
    {        
        Args1* arg = new Args1;
        arg->arr = arr;
        arg->left = i;
        arg->right = std::min((i + RUN - 1), (n - 1));
        // insertionSort((void*)(&arg));
        if (j < threadCount) {
            pthread_create(&tid[j++], nullptr, insertionSort, (void*)(arg));
        } else {
            while (j > 0) {
                pthread_join(tid[--j], NULL);
            }
            ++j;
            pthread_create(&tid[j++], nullptr, insertionSort, (void*)(arg));
        }
    }

    while (j > 0) {
        pthread_join(tid[--j], NULL);
    }

    std::vector<pthread_t> tid2(threadCount);
    j = 0;
    for (int size = RUN; size < n; size = 2 * size) {
        for (int left = 0; left < n; left += 2 * size) {
            int mid = left + size - 1;
            int right = std::min((left + 2 * size - 1), (n - 1));
            if (mid < right) {
                Args2* arg = new Args2;
                arg->arr = arr;
                arg->l = left;
                arg->m = mid;
                arg->r = right;
                // merge((void*)arg);
                if (j < threadCount) {
                    pthread_create(&tid2[j++], nullptr, merge, (void*)(arg));
                } else {
                    while (j > 0) {
                        pthread_join(tid2[--j], NULL);
                    }
                    pthread_create(&tid2[j++], nullptr, merge, (void*)(arg));
                }
            }
        }
    }

    while (j > 0) {
        pthread_join(tid2[--j], NULL);
    }
}    
\end{lstlisting}

\newpage
\section{Тесты}
\begin{lstlisting}[language=C++]
#include <iostream>
#include <algorithm>
#include <gtest/gtest.h>

#include "timsort.hpp"


void TestParent(int* base_arr) {
    int n = sizeof(base_arr) / sizeof(base_arr[0]);
    int arr_1[n], arr_2[n];
    std::copy(base_arr, base_arr + n, arr_1);
    std::copy(base_arr, base_arr + n, arr_2);

    timSort(arr_1, n, 5);
    std::sort(arr_2, arr_2 + n);

    for (int i = 0; i < n; ++i) {
        ASSERT_EQ(arr_1[i], arr_2[i]);
    }
}

TEST(SortTest, ONE) {
    int arr[] = { -2, 7, 15, -14, 0, 15, 0, 7, 234, 3, -89, 9, -12, 21,
                    -7, -4, -13, 5, 8, -14, 12, 54, 1, 0, 127, 40, 55};
    TestParent(arr);
}

TEST(SortTest, TWO) {
    int arr[] = {1, 2, 3, 4, 5, 6, 7, 8, 9, 10, 11, 12, 13, 14, 15, 16,
                    17, 18, 19, 20, 21, 22, 23, 24, 25, 26,};
    TestParent(arr);
}

TEST(SortTest, THREE) {
    int arr[] = {1, 2, -34, 45, -66, 2, 96, -9, 33, 1, 1, 90, 44, -34, -3, 4, 35, 0};
    TestParent(arr);
}
\end{lstlisting}

\newpage
\section{Запуск тестов}
\begin{verbatim}
andrew@DESKTOP-K3DH39N:~/MAI/osLabs/build/tests$ ./lab2_test 
Running main() from /home/andrew/MAI/osLabs/build/_deps/googletest-src/googletest/src/gtest_main.cc
[==========] Running 3 tests from 1 test suite.
[----------] Global test environment set-up.
[----------] 3 tests from SortTest
[ RUN      ] SortTest.ONE
[       OK ] SortTest.ONE (150 ms)
[ RUN      ] SortTest.TWO
[       OK ] SortTest.TWO (0 ms)
[ RUN      ] SortTest.THREE
[       OK ] SortTest.THREE (0 ms)
[----------] 3 tests from SortTest (150 ms total)

[----------] Global test environment tear-down
[==========] 3 tests from 1 test suite ran. (187 ms total)
[  PASSED  ] 3 tests.
\end{verbatim}
\newpage
\section{Выводы}
В результате выполнения данной лабораторной работы была написана программа на языке С++ для сортировки массива методом TimSort, обрабатывающая данные в многопоточном режиме. Были получены практические навыки в управлении потоками в ОС и обеспечении синхронизации между потоками.
\end{document}
