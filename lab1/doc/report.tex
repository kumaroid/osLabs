\documentclass[a4paper, 12pt]{article}
\usepackage{cmap}
\usepackage[12pt]{extsizes}			
\usepackage{mathtext} 				
\usepackage[T2A]{fontenc}			
\usepackage[utf8]{inputenc}			
\usepackage[english,russian]{babel}
\usepackage{setspace}
\singlespacing
\usepackage{amsmath,amsfonts,amssymb,amsthm,mathtools}
\usepackage{fancyhdr}
\usepackage{soulutf8}
\usepackage{euscript}
\usepackage{mathrsfs}
\usepackage{listings}

\usepackage[colorlinks=true, urlcolor=blue, linkcolor=black]{hyperref}

\pagestyle{fancy}
\usepackage{indentfirst}
\usepackage[top=10mm]{geometry}
\rhead{}
\lhead{}
\renewcommand{\headrulewidth}{0mm}
\usepackage{tocloft}
\renewcommand{\cftsecleader}{\cftdotfill{\cftdotsep}}
\usepackage[dvipsnames]{xcolor}

\lstdefinestyle{mystyle}{ 
	keywordstyle=\color{OliveGreen},
	numberstyle=\tiny\color{Gray},
	stringstyle=\color{BurntOrange},
	basicstyle=\ttfamily\footnotesize,
	breakatwhitespace=false,         
	breaklines=true,                 
	captionpos=b,                    
	keepspaces=true,                 
	numbers=left,                    
	numbersep=5pt,                  
	showspaces=false,                
	showstringspaces=false,
	showtabs=false,                  
	tabsize=2
}

\lstset{style=mystyle}

\begin{document}
\thispagestyle{empty}	
\begin{center}
	Московский авиационный институт
	
	(Национальный исследовательский университет)
	
	Факультет "Информационные технологии и прикладная математика"
	
	Кафедра "Вычислительная математика и программирование"
	
\end{center}
\vspace{40ex}
\begin{center}
	\textbf{\large{Лабораторная работа №1 по курсу\linebreak \textquotedblleft Операционные системы\textquotedblright}}
\end{center}
\vspace{35ex}
\begin{flushright}
	\textit{Студент: } Иванов Андрей Кириллович
	
	\vspace{2ex}
	\textit{Группа: } М8О-208Б-22
	
	\vspace{2ex}
	\textit{Преподаватель: } Миронов Евгений Сергеевич
	
	\vspace{2ex}
	\textit{Вариант: } 13
	
	\vspace{2ex}
	\textit{Оценка: } \underline{\quad\quad\quad\quad\quad\quad}
	
	 \vspace{2ex}
	\textit{Дата: } \underline{\quad\quad\quad\quad\quad\quad}
	
	\vspace{2ex}
	\textit{Подпись: } \underline{\quad\quad\quad\quad\quad\quad}
	
\end{flushright}

\vspace{5ex}

\begin{vfill}
	\begin{center}
		Москва, 2023
	\end{center}	
\end{vfill}
\newpage


\begingroup
\color{black}
\tableofcontents\newpage
\endgroup

\section{Репозиторий}
\href{https://github.com/kumaroid/osLabs}{https://github.com/kumaroid/osLabs}

\section{Цель работы}
Приобретение практических навыков в:
\begin{itemize}
  \item Управлении процессами в ОС
  \item Обеспечении обмена данных между процессами посредством каналов
\end{itemize}

\section{Задание}
Составить и отладить программу на языке Си, осуществляющую работу с процессами и 
взаимодействие между ними в одной из двух операционных систем. В результате работы 
программа (основной процесс) должен создать для решение задачи один или несколько 
дочерних процессов. Взаимодействие между процессами осуществляется через системные 
сигналы/события и/или каналы (pipe).
Необходимо обрабатывать системные ошибки, которые могут возникнуть в результате работы.

\section{Описание работы программы}
Родительский процесс создает два дочерних процесса. Перенаправление стандартных потоков
ввода-вывода показано на картинке выше. Child1 и Child2 можно «соединить» между собой
дополнительным каналом. Родительский и дочерний процесс должны быть представлены
разными программами.
Родительский процесс принимает от пользователя строки произвольной длины и пересылает их в
pipe1. Процесс child1 и child2 производят работу над строками. Child2 пересылает результат своей
работы родительскому процессу. Родительский процесс полученный результат выводит в
стандартный поток вывода.

В ходе выполнения лабораторной работы я использовал следующие системные вызовы:
\begin{itemize}
  \item fork() - создание нового процесса
  \item pipe() - создание канала
  \item dup2() - создание копии файлового дескриптора, используя для нового дескриптора самый маленький свободный номер файлового дескриптора.
  \item execlp() - запуск файла на исполнение
\end{itemize}


\newpage

\section{Исходный код}
parent.hpp
\begin{lstlisting}[language=C++]
#pragma once

#include <iostream>
#include <fstream>
#include <string>
#include <sys/wait.h>
#include <unistd.h>

int ParentWork();
void CreatePipe(int *fd);
int CreateFork();
void DoClose(int fd);
\end{lstlisting}

parent.cpp
\begin{lstlisting}[language=C++]
#include "parent.hpp"

void CreatePipe(int *fd) {
    int status;
    status = pipe(fd);
    if (status == -1) {
        std::cerr << "Failed pipe()\n";
        exit(-1);
    }
}

int CreateFork() {
    int pid;
    pid = fork();
    if (pid == -1) {
        std::cerr << "Failed fork()\n";
        exit(-2);
    }
    return pid;
}

void DoClose(int fd) {
    int status;
    status = close(fd);
    if (status == -1) {
        std::cerr << "Failed close()\n";
        exit(-3);
    }
}

void DoDup(int a, int b) {
    int status;
    status = dup2(a, b);
    if (status == -1) {
        std::cerr << "Failed dup2()\n";
        exit(-4);
    }
}

int ParentWork() {
    pid_t pid;
    int status;
    int capacity = 0;
    int fd1[2];
    int fd2[2];
    int connect[2];
    char *child1;
    char *child2;
    std::string line;

    child1 = getenv("PATH_CHILD1");
    child2 = getenv("PATH_CHILD2");

    CreatePipe(fd1);
    CreatePipe(fd2);
    CreatePipe(connect);

    pid = CreateFork();
    if (pid == 0) { // child 1
        DoClose(connect[0]);
        DoClose(fd1[1]);
        DoDup(connect[1], STDOUT_FILENO);
        if (execlp(child1, "lower.out", std::to_string(fd1[0]).c_str(), nullptr) == -1) {
            std::cerr << "Failed execlp()\n";
            exit(-5);
        }
    }
    pid = CreateFork();
    if (pid == 0) { // child 2
        DoClose(connect[1]);
        DoClose(fd2[0]);
        DoDup(connect[0], STDIN_FILENO);
        if (execlp(child2, "underscore.out", std::to_string(fd2[1]).c_str(), nullptr) == -1) {
            std::cerr << "Failed execlp()\n";
            exit(-5);
        }
    }

    if (pid != 0) {
        DoClose(fd1[0]);
        DoClose(fd2[1]);
        char chWrite;
        char end = '\0';
        char toOut[256];
        int readCount;
        while (std::getline(std::cin, line)) {
            write(fd1[1], line.c_str(), line.size());

            chWrite = '\n';
            write(fd1[1], &chWrite, sizeof(chWrite));

            capacity += line.size() + 1;
        }
        write(fd1[1], &end, sizeof(end));

        while(capacity > 0) {
            readCount = read(fd2[0], toOut, sizeof(toOut));
            if (readCount != -1) {
                capacity -= readCount;
            }
            //std::cout << toOut << std::flush;
            for (int i = 0; i < readCount; ++i) {
                std::cout << toOut[i] << std::flush;
            }   
        }
    }

    waitpid(-1, &status, 0);
    waitpid(-1, &status, 0);
    return 0;
}

\end{lstlisting}

lower.cpp
\begin{lstlisting}[language=C++]
#include <iostream>
#include <string>
#include <unistd.h>
#include <sys/wait.h>
#include <sys/stat.h>
#include <fcntl.h>

int main (int argc, char** argv) {
    if (argc != 2) {
        std::cerr << "Wrong argc in lower.out\n";
        exit(-6);
    }
    char chLow;
    int fd = std::atoi(argv[1]);
    while(true) {
        read(fd, &chLow, sizeof(chLow));
        if(chLow == '\0') {
            break;
        }
        chLow = (char)tolower(chLow);
        std::cout << chLow << std::flush;
    }
    std::cout << '\0';
}
\end{lstlisting}

underscore.cpp
\begin{lstlisting}[language=C++]
#include <iostream>
#include <string>
#include <unistd.h>
#include <sys/wait.h>
#include <sys/stat.h>
#include <fcntl.h>

int main (int argc, char** argv) {
    if (argc != 2) {
        std::cerr << "Wrong argc in underscore.out\n";
        exit(-6);
    }
    char ch;
    int fd = std::atoi(argv[1]);
    (void)argc;
    while(true) {
        std::cin.get(ch);
        if (ch == ' ') {
            ch = '_';
        } else if (ch == '\0') {
            break;
        }
        write(fd, &ch, sizeof(ch));
    }
}

\end{lstlisting}

parent.cpp
\begin{lstlisting}[language=C++]
#include "parent.hpp"

void parentProcess(const char *pathToChild) {
    std::string fileName;
    getline(std::cin, fileName);

    int fd1[2], fd2[2];
    createPipe(fd1);
    createPipe(fd2);
    
    int pid = createChildProcess();
    if (pid != 0) { // Parent process
        close(fd1[0]);
        close(fd2[1]);
        
        std::string str;
        while (getline(std::cin, str)) {
            str += "\n";
            write(fd1[1], str.c_str(), str.length()); // from str to fd1[1]
        }
        close(fd1[1]);

        std::stringstream output = readFromPipe(fd2[0]);
        while(std::getline(output, str)) {
            std::cout << str << std::endl;
        }
        close(fd2[0]);
    } else { // Child process
        close(fd1[1]);
        close(fd2[0]);

        if (dup2(fd1[0], STDIN_FILENO) == -1 || dup2(fd2[1], STDOUT_FILENO) == -1) {
            perror("Error with dup2");
            exit(EXIT_FAILURE);
        }
        
        if (execlp(pathToChild, pathToChild, fileName.c_str(), nullptr) == -1) { // to child.cpp
            perror("Error with execlp");
            exit(EXIT_FAILURE);
        } 
    }
}
\end{lstlisting}

main.cpp
\begin{lstlisting}[language=C++]
#include "parent.hpp"

int main() {
    ParentWork();
    return 0;
}
\end{lstlisting}

\newpage
\section{Тесты}

\begin{lstlisting}[language=C++]
#include <iostream>
#include <gtest/gtest.h>

#include "parent.hpp"

void TestParent(std::string &src, std::string &res) {
    std::istringstream srcStream(src);

    std::streambuf* buf = std::cin.rdbuf(srcStream.rdbuf());

    testing::internal::CaptureStdout();
    ParentWork(); 
    ASSERT_EQ(testing::internal::GetCapturedStdout(), res + '\n'); 

    std::cin.rdbuf(buf);
}


TEST(cin_test, ONE) {
    std::string src = "       ";
    std::string res = "_______";
    TestParent(src, res);
}

TEST(cin_test, TWO) {
    std::string src = "AHAHAHAHAH";
    std::string res = "ahahahahah";
    TestParent(src, res);
}

TEST(cin_test, THREE) {
    std::string src = "   HELLO wOrLd 12 3";
    std::string res = "___hello_world_12_3";
    TestParent(src, res);
}

\end{lstlisting}

\newpage
\section{Запуск тестов}
\begin{verbatim}
roman@DESKTOP-K3DH39N:~/MAI/MAI_OS_labs/build/tests/lab1_test$ ./lab1_test 
Running main() from /home/roman/MAI/MAI_OS_labs/build/_deps/googletest-src/googletest/src/gtest_main.cc
[==========] Running 3 tests from 1 test suite.
[----------] Global test environment set-up.
[----------] 3 tests from cin_test
[ RUN      ] cin_test.ONE
[       OK ] cin_test.ONE (1 ms)
[ RUN      ] cin_test.TWO
[       OK ] cin_test.TWO (1 ms)
[ RUN      ] cin_test.THREE
[       OK ] cin_test.THREE (1 ms)
[----------] 3 tests from cin_test (3 ms total)

[----------] Global test environment tear-down
[==========] 3 tests from 1 test suite ran. (3 ms total)
[  PASSED  ] 3 tests.
\end{verbatim}
\newpage
\section{Выводы}

В результате выполнения данной лабораторной работы была написана программа на языке С++, осуществляющая работу с процессами и 
взаимодействие между ними. Были приобретены практические навыки в управлении процессами в ОС и обеспечении обмена данных между процессами посредством каналов.
\end{document}
