\documentclass[a4paper, 12pt]{article}
\usepackage{cmap}
\usepackage[12pt]{extsizes}			
\usepackage{mathtext} 				
\usepackage[T2A]{fontenc}			
\usepackage[utf8]{inputenc}			
\usepackage[english,russian]{babel}
\usepackage{setspace}
\singlespacing
\usepackage{amsmath,amsfonts,amssymb,amsthm,mathtools}
\usepackage{fancyhdr}
\usepackage{soulutf8}
\usepackage{euscript}
\usepackage{mathrsfs}
\usepackage{listings}

\usepackage[colorlinks=true, urlcolor=blue, linkcolor=black]{hyperref}

\pagestyle{fancy}
\usepackage{indentfirst}
\usepackage[top=10mm]{geometry}
\rhead{}
\lhead{}
\renewcommand{\headrulewidth}{0mm}
\usepackage{tocloft}
\renewcommand{\cftsecleader}{\cftdotfill{\cftdotsep}}
\usepackage[dvipsnames]{xcolor}

\lstdefinestyle{mystyle}{ 
	keywordstyle=\color{OliveGreen},
	numberstyle=\tiny\color{Gray},
	stringstyle=\color{BurntOrange},
	basicstyle=\ttfamily\footnotesize,
	breakatwhitespace=false,         
	breaklines=true,                 
	captionpos=b,                    
	keepspaces=true,                 
	numbers=left,                    
	numbersep=5pt,                  
	showspaces=false,                
	showstringspaces=false,
	showtabs=false,                  
	tabsize=2
}

\lstset{style=mystyle}

\begin{document}
\thispagestyle{empty}	
\begin{center}
	Московский авиационный институт
	
	(Национальный исследовательский университет)
	
	Факультет "Информационные технологии и прикладная математика"
	
	Кафедра "Вычислительная математика и программирование"
	
\end{center}
\vspace{40ex}
\begin{center}
	\textbf{\large{Лабораторная работа №2 по курсу\linebreak \textquotedblleft Операционные системы\textquotedblright}}
\end{center}
\vspace{35ex}
\begin{flushright}
	\textit{Студент: } Иванов Андрей Кириллович
	
	\vspace{2ex}
	\textit{Группа: } М8О-208Б-22
	
	\vspace{2ex}
	\textit{Преподаватель: } Миронов Евгений Сергеевич
	
	\vspace{2ex}
	\textit{Вариант: } 13
	
	\vspace{2ex}
	\textit{Оценка: } \underline{\quad\quad\quad\quad\quad\quad}
	
	 \vspace{2ex}
	\textit{Дата: } \underline{\quad\quad\quad\quad\quad\quad}
	
	\vspace{2ex}
	\textit{Подпись: } \underline{\quad\quad\quad\quad\quad\quad}
	
\end{flushright}

\vspace{5ex}

\begin{vfill}
	\begin{center}
		Москва, 2023
	\end{center}	
\end{vfill}
\newpage


\begingroup
\color{black}
\tableofcontents\newpage
\endgroup

\section{Репозиторий}
\href{https://github.com/kumaroid/osLabs}{https://github.com/kumaroid/osLabs}

\section{Цель работы}
Приобретение практических навыков в:
\begin{itemize}
    \item Освоении принципов работы с файловыми системами
    \item Обеспечении обмена данных между процессами посредством технологии «File mapping»
\end{itemize}

\section{Задание}
Составить и отладить программу на языке Си, осуществляющую работу с процессами и взаимодействие между ними в одной из двух операционных систем. В результате работы программа (основной процесс) должен создать для решение задачи один или несколько дочерних процессов. Взаимодействие между процессами осуществляется через системные сигналы/события и/или через отображаемые файлы (memory-mapped files).
Необходимо обрабатывать системные ошибки, которые могут возникнуть в результате работы.

\section{Описание работы программы}
Задание аналогично первой лабораторной работе.

В ходе выполнения лабораторной работы я использовал следующие системные вызовы:
\begin{itemize}
    \item fork() - создание нового процесса
    \item sem\_open() - создание/открытие семафора
    \item sem\_post() - увеличивание значения семафора и разблокировка ожидающих потоков
    \item sem\_wait() - уменьшение значения семафора. Если 0, то вызывающий поток блокируется 
    \item sem\_close() - закрытие семафора
    \item shm\_open() - создание/открытие разделяемой памяти POSIX
    \item shm\_unlink() - закрытие разделяемой памяти
    \item ftruncate() - уменьшение длины файла до указанной  
    \item mmap() - отражение файла или устройства в памяти
    \item munmap() - снятие отражения
    \item execlp() - запуск файла на исполнение
\end{itemize}


\newpage

\section{Исходный код}
utils.hpp
\begin{lstlisting}[language=C++]
#pragma once

#include <iostream>
#include <fstream>
#include <string>
#include <unistd.h>
#include <sys/wait.h>
#include <sys/stat.h>
#include <fcntl.h>
#include <string.h>
#include <sys/mman.h>
#include <sys/stat.h>     
#include <fcntl.h>
#include <unistd.h>
#include <semaphore.h>

sem_t* CreateSemaphore(const char *name, int value);
int CreateShm(const char* name);
char* MapSharedMemory(const int size, int fd);
int CreateFork();

constexpr const char *SEM_1 = "SEM_1";
constexpr const char *SEM_2 = "SEM_2";
constexpr const char *SEM_3 = "SEM_3";

const int FILE_SIZE = 1024;

\end{lstlisting}

utils.cpp
\begin{lstlisting}[language=C++]
#include "utils.hpp"

sem_t* CreateSemaphore(const char *name, int value) {
    sem_t *semptr = sem_open(name, O_CREAT, S_IRUSR | S_IWUSR, value);
    if (semptr == SEM_FAILED){
        perror("Couldn't open the semaphore");
        exit(EXIT_FAILURE);
    }
    return semptr;   
}

int CreateShm(const char* name) {
    int fd = shm_open(name, O_CREAT | O_RDWR, S_IRUSR | S_IWUSR);
    if (fd == -1) {
        std::cerr << "Failed shm_open\n";
        exit(-1);
    }
    ftruncate(fd, 1024);
    return fd;
}

char* MapSharedMemory(const int size, int fd) {
    char *memptr = (char*)mmap(nullptr, size, PROT_READ | PROT_WRITE, MAP_SHARED, fd, 0);
    if (memptr == MAP_FAILED) {
        perror("Error with file mapping");
        exit(EXIT_FAILURE);
    }
    return memptr;
}

int CreateFork() {
    int pid;
    pid = fork();
    if (pid == -1) {
        std::cerr << "Failed fork()\n";
        exit(-2);
    }
    return pid;
}
\end{lstlisting}

parent.hpp
\begin{lstlisting}[language=C++]
#pragma once


#include "utils.hpp"

int ParentWork();
\end{lstlisting}

parent.сpp
\begin{lstlisting}[language=C++]
#include "parent.hpp"
#include "utils.hpp"

int ParentWork() {
    pid_t pid;
    int status;;
    char *child1;
    char *child2;
    std::string line;

    char* name1 = "/shm1";
    sem_t *semptr1 = CreateSemaphore(SEM_1, 0);
    int shd_fd1 = CreateShm(name1);
    char *memptr1 = MapSharedMemory(FILE_SIZE, shd_fd1);

    char* name2 = "/shm2";
    sem_t *semptr2 = CreateSemaphore(SEM_2, 0);
    int shd_fd2 = CreateShm(name2);
    char *memptr2 = MapSharedMemory(FILE_SIZE, shd_fd2);

    char* name3 = "/shm3";
    sem_t *semptr3 = CreateSemaphore(SEM_3, 0);
    int shd_fd3 = CreateShm(name3);
    char *memptr3 = MapSharedMemory(FILE_SIZE, shd_fd3);

    child1 = getenv("PATH_CHILD1");
    child2 = getenv("PATH_CHILD2");

    pid = CreateFork();
    if (pid == 0) { // child 1
        if (execlp(child1, "lower.out", name1, name2, nullptr) == -1) {
            std::cerr << "Failed execlp()\n";
            exit(-5);
        }
    }

    pid = CreateFork();
    if (pid == 0) { // child 2
        if (execlp(child2, "underscore.out", name2, name3, nullptr) == -1) {
            std::cerr << "Failed execlp()\n";
            exit(-5);
        }
    }

    if (pid != 0) {
        while (std::getline(std::cin, line)) {
            line += '\n';
            strcpy(memptr1, line.c_str());
            sem_post(semptr1);

            if (line == "\n") {
                strcpy(memptr1, "\0");
                sem_post(semptr1);
                break;
            }
            sem_wait(semptr3);
            std::string_view st(memptr3);
            std::string s = {st.begin(), st.end()};
            if (s != "\n") {
                std::cout << s << std::endl;
                strcpy(memptr3, "\n");
            }
        }
    }

    waitpid(-1, &status, 0);
    waitpid(-1, &status, 0);

    sem_close(semptr1);
    sem_unlink(SEM_1);
    shm_unlink("/shm1");
    munmap(memptr1, FILE_SIZE);
    close(shd_fd1);
    
    sem_close(semptr2);
    sem_unlink(SEM_2);
    shm_unlink("/shm2");
    munmap(memptr2, FILE_SIZE);
    close(shd_fd2);

    sem_close(semptr3);
    sem_unlink(SEM_3);
    shm_unlink("/shm3");
    munmap(memptr3, FILE_SIZE);
    close(shd_fd3);
    return 0;
}

\end{lstlisting}

underscore.сpp
\begin{lstlisting}[language=C++]
#include "utils.hpp"

int main (int argc, char** argv) {
    if (argc != 3) {
        std::cerr << "Wrong argc in underscore.out\n";
        exit(-6);
    }

    const char* fileName = argv[1];
    const char* fileName2 = argv[2];

    sem_t *semptr1 = CreateSemaphore(SEM_2, 0);
    int shd_fd1 = CreateShm(fileName);
    char *memptr1 = MapSharedMemory(FILE_SIZE, shd_fd1);

    sem_t *semptr2 = CreateSemaphore(SEM_3, 0);
    int shd_fd2 = CreateShm(fileName2);
    char *memptr2 = MapSharedMemory(FILE_SIZE, shd_fd2);

    while(true) {
        sem_wait(semptr1);
        std::string_view st(memptr1);
        std::string s = {st.begin(), st.end()};
        for (int i = 0; i < s.size(); ++i) {
            if (s[i] == ' ') {
                s[i] = '_';
            }
        }
        
        if (s != "\n")  {
            strcpy(memptr2, s.c_str());
            strcpy(memptr1, "\n");
            sem_post(semptr2);
        }

        if (s == "\0") {
            break;
        }
    }

    sem_close(semptr1);
    sem_unlink(SEM_2);
    shm_unlink(fileName);
    munmap(memptr1, FILE_SIZE);
    close(shd_fd1);

    sem_close(semptr2);
    sem_unlink(SEM_3);
    shm_unlink(fileName2);
    munmap(memptr2, FILE_SIZE);
    close(shd_fd2);

}
    
\end{lstlisting}

\newpage
\section{Тесты}
\begin{lstlisting}[language=C++]
#include <iostream>
#include <gtest/gtest.h>

#include "parent.hpp"

void TestParent(std::string &src, std::string &res) {
    std::istringstream srcStream(src);

    std::streambuf* buf = std::cin.rdbuf(srcStream.rdbuf());

    testing::internal::CaptureStdout();
    ParentWork(); 
    ASSERT_EQ(testing::internal::GetCapturedStdout(), res + '\n'); 

    std::cin.rdbuf(buf);
}


TEST(cin_test, ONE) {
    std::string src = "       \n\n";
    std::string res = "_______\n";
    TestParent(src, res);
}

TEST(cin_test, TWO) {
    std::string src = "AHAHAHAHAH\n\n";
    std::string res = "ahahahahah\n";
    TestParent(src, res);
}

TEST(cin_test, THREE) {
    std::string src = "   HELLO wOrLd 12 3\n\n";
    std::string res = "___hello_world_12_3\n";
    TestParent(src, res);
}

\end{lstlisting}

\newpage
\section{Запуск тестов}
\begin{verbatim}
andrew@DESKTOP-K3DH39N:~/MAI/osLabs/build/tests$ ./lab3_test 
Running main() from /home/andrew/MAI/osLabs/build/_deps/googletest-src/googletest/src/gtest_main.cc
[==========] Running 3 tests from 1 test suite.
[----------] Global test environment set-up.
[----------] 3 tests from cin_test
[ RUN      ] cin_test.ONE
[       OK ] cin_test.ONE (14 ms)
[ RUN      ] cin_test.TWO
[       OK ] cin_test.TWO (1 ms)
[ RUN      ] cin_test.THREE
[       OK ] cin_test.THREE (1 ms)
[----------] 3 tests from cin_test (16 ms total)

[----------] Global test environment tear-down
[==========] 3 tests from 1 test suite ran. (16 ms total)
[  PASSED  ] 3 tests.
\end{verbatim}

\newpage
\section{Выводы}
В результате выполнения данной лабораторной работы была написана программа на языке С++, осуществляющая работу с процессами и взаимодействие между ними через системные сигналы и отображаемые файлы. Были приобретены практические навыки в освоении принципов работы с файловыми системами и обеспечении обмена данных между процессами посредством технологии «File mapping».\end{document}
